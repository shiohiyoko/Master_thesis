%%
%% This is file `minimal_imlex.en.tex',
%% generated with the docstrip utility.
%%
%% The original source files were:
%%
%% uefcsthesis.dtx  (with options: `ex,en,modern,imlex')
%% 
%% This is a generated file.
%% 
%% Copyright (C) 2018--2022 by Pauli Miettinen <pauli.miettinen@uef.fi>
%% 
%% This file may be distributed and/or modified under the conditions of
%% the LaTeX Project Public License, either version 1.3c of this license
%% or (at your option) any later version.  The latest version of this
%% license is in:
%% 
%% http://www.latex-project.org/lppl.txt
%% 
%% and version 1.3c or later is part of all distributions of LaTeX
%% version 2006/05/20 or later.
%% 
%% 
%% This is a minimal example of using the uefcsthesis class.
%% This generates an English MSc thesis with one-sided layout.
%% To compile, use either lualatex or xelatex, for example,
%% $ lualatex minimal_modern.en.tex
%% $ biber minimal_modern.en
%% $ lualatex minimal_modern.en.tex
%% or use latexmk:
%% $ latexmk -lualatex minimal_modern.en.tex
%%
%% When returning the final thesis to library, you must also return the source code
%% used to create the final version. This is easiest if you flatten your document into
%% a single file (+ image files). To do that, use can use latexpand (see
%% https://www.ctan.org/pkg/latexpand), which is also part of TeX Live and
%% MiKTeX packages. An example use of latexpand would be
%% $ lualatex minimal_modern.en.tex
%% $ biber minimal_modern.en
%% $ latexpand --empty-comments --biber minimal_modern.en.bbl \
%% > minimal_modern.en.tex > flat_thesis.tex
%% $ lualatex flat_thesis
%% $ lualatex flat_thesis
%% which produces files called flat_thesis.tex and flat_thesis.pdf that can be
%% returned to the library.
%%

\documentclass[mscthesis,english,oneside,biblatex,imlex]{template/uefcsthesis}
%% Correct the below with the name of your bibliography file
\addbibresource{bibliography.bib}
\usepackage{txfonts}
\usepackage{amsmath}

% Fabiano: these are packages added by me:
\usepackage[dvipsnames]{xcolor}  % enables textcolor
\usepackage[acronym, toc]{glossaries}  % enables glossaries
\usepackage{lastpage}
\usepackage{fancyhdr}


%% Replace all capital text with your own information.
\title{Text-based Person re-Identification for human tracking} % Title of the thesis
\author{Yuya}{Takagi} % Your name
\date{\thismonth} % The month and year of handing in your thesis, or \thismonth of automatic
\city{Joensuu} % Either Kuopio or Joensuu
\firstsupervisor{Jun Miura} % Name of the first supervisor
\keywords{ Text-based person Re-identification\sep Textt-based person retrieval} % Keywords must be separated with \sep

%% To get the ACM CCS classification, you can visit
%% https://dl.acm.org/ccs/ccs.cfm
%% There you can find a tool to generate LaTeX code for the classification
%% Copy it here. You don't need to copy the XML at the begin, though.
%% For example,
%% \ccsdesc[500]{Some Class}

%% You can change department, faculty, and university names using the
%% following commands (for now commented out)
%% \setstring{universityname}{Toyohashi University of Technology}
%% \setstring{facultyname}{Faculty of Robotics}
%% \setstring{departmentname}{Department of Computer Science}
%% If you need to use different languages, use \setstring[lang]{<stringname>}{<text>}.
%% You can use \setstring to add translations to all strings in title and abstract
%% pages, see uefcsthesis.pdf, section ``Pre-Defined Strings and a Macro to Change
%% Them'' for further information.


\makenoidxglossaries
\newglossaryentry{maths}
{
    name=mathematics,
    description={Mathematics is what mathematicians do}
}

\begin{document}
\maketitle

% \begin{abstract}
% Person re-identification is used to 
    
% \end{abstract}

\frontmatter
\tableofcontents
\mainmatter



% Each chapter is separated into a different .tex file. Include them here.
\chapter{Introduction}
\label{cha:Introduction}

Simultaneous Localization and Mapping (SLAM) using LiDAR technology stands as a cornerstone in autonomous navigation systems, enabling real-time mapping and localization essential for the robust and safe operation of various autonomous platforms. However, difficulties arise in mapping environments that contain specular or transparent surfaces as laser rays' reflections and transmissions on these areas lead to inaccuracies in the generated map, posing potential hazards during navigation. Recent research has introduced several novel techniques aimed at addressing this issue, but these approaches often exhibit constraints: some rely on incident angles closely aligned to normal; others are limited by specialized material handling, lacking adaptability across diverse surfaces; certain algorithms struggle with real-time processing, impeding their practical application.

\section{Motivation}

\section{Objectives}

\section{Stakeholders}

\section{Document structure}
\chapter{Theoretical Background}

\section{LiDAR}

\section{SLAM}

\textcolor{red}{[WIP] Other sections that are relevant to understanding the content, e.g. sensor fusion, intensity, AI. Should the glass detection techniques be explained here already, or in the literature review section?}
\chapter{Related Studies }
{\color{red} need to think about this a little further}
To provide a comprehensive understanding of the context and foundation for this research, we review related studies that have explored multi-modal representation and the impact of masking ratios in Vision-Language Models.

% ------------------------------------------------------------ %
{\color{red}maybe not required}
\section{Research Methodology}
We first delve into the existing literature, drawing from papers accessible through the "Paper with Code" platform. This platform typically provides research papers along with their associated code implementations. {\color{red}This papers was selected based on its prominence, measured by the reported accuracy and success in the field, as stated in the platform.}

Following the identification of the primary paper, the researcher conducts a thorough review of its content, focusing particularly on aspects related to methodology. This involves understanding the proposed techniques, algorithms, and approaches presented in the paper to achieve high accuracy in the context of text-based person searches. The aim is to comprehend the nuances of the existing methodology and identify the key factors contributing to its success.

In addition to the primary paper, the researcher examines two other papers that exhibit a significant difference in accuracy. This comparative analysis is valuable for gaining insights into different approaches within the field. We aim to investigate diverse perspectives and approaches, especially if there is a notable contrast in reported accuracy metrics.
We scrutinizes the methodologies of these selected papers, comparing and contrasting them with the primary paper. This comparative analysis helps identify the strengths and weaknesses of different approaches, shedding light on potential areas of improvement or innovation for the current research.

Overall, the methodology involves a comprehensive exploration of relevant literature, with a focus on the primary paper selected from "Papers with Code." The intent is to understand the methodologies employed in achieving high accuracies and to leverage insights from other papers with varying performance metrics. 

However, if only Papers with Code is used, the information obtained is limited and biased. To eliminate this bias, we decided to use the Scopus database to search for a wider range of papers by keyword search.

\subsection*{Identification}
From this research question, two main keywords that sufficiently explain the topic were used: person retrieval and vision language pre-training.
Furthermore, synonyms and related terms were associated to these keywords to form keyword groups as follows:

\begin{itemize}
    \item person:
    \begin{itemize}
        \item retrieval;
        \item detection;
        \item search.
    \end{itemize}
    \item vision language pre-training:
    \begin{itemize}
        \item VLP;
        \item text.
    \end{itemize}
\end{itemize}


From the keywords, we had a keyword search on scopus from the search strings as follows:

\begin{itemize}
    \item ( "person" AND ("retrieval" OR "detection" OR "search") ) AND ( "vision language pre-training" OR "VLP" OR "text" ).
\end{itemize}

The Scopus search yielded a total of $20170$ documents. Within this result, we set the subject area to Computer Science, document type to article and conference paper, language to only english, and set the open access to all open access. With this filters, $862$ articles were found. 

\subsection{Screening}

Various factors were taken into account for the exclusion of documents:
{\color{red} changes required}
\begin{enumerate}
    \item problem and goal were too different (e.g., building new hardware, analysis of leaf reflectance);
    \item not sufficiently related to this work (e.g., focused on hyperspectral );
    \item duplicates that were not automatically detected and excluded.
\end{enumerate}

% ------------------------------------------------------------ %

\section{Vision-language Models}
Vision-language models have been researched significantly by leveraging end-to-end trainable deep neural networks (DNN). Before the vision-language model, both vision and language were researched with different DNN model structures. 

The field of vision recognition has been rapidly evolving in recent years. Until now, the vision recognition models have experienced 5 stages (\cite{zhang2024visionlanguagemodelsvisiontasks}): traditional machine learning, deep learning from scratch, supervised pre-training with fine-tuning, unsupervised pre-training with fine-tuning, and vision-language model with pre-training.

Traditional machine learning relies on feature engineering. To achieve features, general methods use hand-craft features (\cite{svmclassification}) and lightweight models (\cite{knn, svm}) to classify images into predefined categories. However, this method demands domain experts to design effective features for specific visual recognition tasks, which makes it ineffective for complex tasks and limits its scalability.

To overcome the problems with traditional methods, deep learning methods were devised (\cite{imagenet, dnn_imagerecognition}) to enhance feature engineering and allowed to focus on the architecture engineering of neural networks to learn features effectively. Great success came from ResNet (\cite{resnet}) with the introduction of residual connections as a solution to the gradient vanishing problem, achieving unprecedented performance.However, this approach faced issues such as slow training convergence and the need for extensive labeled data. 

Recent studies (\cite{radford2021learning}) have discovered that the features learned from large-scale labeled datasets possess a high degree of versatility and can be effectively transferred to a variety of downstream tasks. This indicates that the patterns and representations captured during the training on these extensive datasets are not only specific to the initial context but are also generalizable, allowing them to be utilized in different, often more specialized applications. The transferability underscores the potential of leveraging pre-trained models to enhance performance and efficiency in a wide range of tasks across various domains. 
These learning techniques accelerates training models with limited task-specific training datasets while achieving high performance.

{\color{red}supervision}

Supervised pre-training outperformed many of the previous state-of-the-art performance on visual recognition tasks, yet still has the problem of preparing large-scale labeled datasets. To overcome the dependency on labeled data, \cite{he2020momentumcontrastunsupervisedvisual} presented a method called Momentum Contrast (MoCo), which builds a dynamic dictionary with a queue and a moving-averaged encoder, enabling the model to learn effective representation without labeled data. Other methods are Simple Framework for Contrastive learning (SimCLR) introduced by \cite{chen2020simpleframeworkcontrastivelearning}. This is a simplified contrastive learning framework for visual representation, avoiding specialized architectures or memory banks. Beyond these foundations, pre-training models no longer rely on labeled datasets, which enables to learn from various training data compared to supervised pre-training. 

Drastic improvements have emerged from natural language processing (\cite{devlin2018bert, brown2020language}), combined with previous vision recognition models. Specifically, a new paradigm called vision-language model pre-training has been proposed for vision recognition. To pre-train, this model does not require labeled datasets, but instead image-text pairs. With this, the training datasets can be found infinitely on the internet. However, the image-text pair on the internet contains information that does not correlate to each other. To train the model with noisy data, VLM is trained with certain vision-language objectives (\cite{radford2021learning, yu2022cocacontrastivecaptionersimagetext}). From these objectives, the VLM matches the embedding of any given images and text, which enables the performance of zero-shot prediction without fine-tuning on downstream visual recognition.

% Vision-Language models are a model that combines both the vision and language modalities and enables to process both information. 
% Take, for example, the task of zero-shot image classification. We’ll pass an image and a few prompts like so to obtain the most probable prompt for the input image.
% To predict the probable prompt, the model needs to understand both input image and the text prompts. To understand those modalities, the model will have separate or fused encoders for both vision and language.

\section{Masking Strategies}
Learning robust representation from image and text is the key challenge for VLM. An approach to this challenge is previously done with masked language modeling. This is a pre-training task for language models, where the model learns to predict masked tokens in a sequence based on the surrounding context. This strategy is foundational for several state-of-the-art models (\cite{devlin2018bert, liu2020roberta}). 

In MLM, it is necessary to develop a strategy to mask the text. Key masking strategy are random masking, span-based masking, whole word masking, and entity masking. 

Random masking are done by randomly select the tokens from the sentence and replaces them with special token, [MASK]. For example, in BERT's pre-training, 15\% of the tokens are masked, out of which 80\% are replaced with [MASK], 10\% are replaced with random tokens, and remaining 10\% unchanged. 

To understand longer text spans, span-based masking was conceived. Span-based masking involves masking contiguous spans of tokens, with this, the model will be encouraged to learn dependencies across longer text spans. 

Another method is whole word masking. This strategy masks entire words instead of sub-word units. If any part of a word is selected for masking, the entire word is masked.

In VLM, similar masking strategies can be employed to enhance the model's ability to understand and generate language in relation to visual content.

Text caption masking is a masking strategy to mask the words or phrases in the text caption. It is used in VLM training to reconstruct the masked tokens based on the visual content and the unmasked text. This approach leverages the contextual understanding from both modalities. ImageBERT (\cite{qi2020imagebertcrossmodalpretraininglargescale}) used masked language modeling on text captions to pre-train the VLM.

Another method is visual patch masking: patches of an image are masked and the VLM is trained to reconstruct the missing parts based on the unmasked image regions and the associated text.

\section{Model Used In This Thesis}
In this section, vision language model used in this thesis is introduced.

\subsection{ALBEF: Align before Fuse}
Text encoder and image encoders are used to learn the representation in vision-and-language pre-training models (VLP). Most existing VLP models use multi-modal encoders to jointly model visual tokens (image features) and word tokens (text features). Although this method gives a certain level of performance, it faces challenges if the visual and word tokens are unaligned, making it difficult for the encoder to learn interactions between them. 
To tackle this challenge, \cite{li2021align} first aligns image and text representations before fusing them through cross-modal attentions. 

\begin{figure}[htbp]
    \begin{center}
        \includegraphics[width=\linewidth]{img/albef_model_structure.png}
        \caption{Structure of ALBEF: Souce from \cite{li2021align}}
        \label{fig:albef}
    \end{center}
\end{figure}

The author represented a model with an image encoder, text encoder, and a multi-modal encoder. They use 12-layer visual transformer ViT-B/16 (\cite{dosovitskiy2021image}), pre-trained with ImageNet-1k, as the image encoder. 
Text encoder is pre-trained by the first 6 layers of $BERT_{base}$. Multi-modal encoder, fusing text and image features, is pre-trained with the last 6 layers of $BERT_{base}$.
The text encoder transforms an input text $T$ into a sequence of embedding $\{w_{cls}, w_1, ..., w_N\}$. The image features also transform an input image $I$ into a sequence of embedding $\{v_{cls}, v_1, ..., v_M\}$. The text features are fed to a multi-modal encoder, while the image features are fused through cross attention at each layer of the multi-modal encoder.

In the pre-training task, the model has three objectives: image-text contrastive learning (ITC) on the unimodal encoders, masked language modeling (MLM), and image-text matching (ITM).
\cite{li2021align} utilizes the hard negative mined from ITC to improve ITM.

Image text contrastive learning is used to align the representation from the unimodal encoders before fusing. This objective makes it easier for the multi-modal encoder to perform cross-modal learning, as the input features are already aligned in a common space. 

Masked language modeling (MLM) is utilized with both image and text information to determine the masked text. \cite{li2021align} randomly masks 15\% of the input token with specialized token [MASK], which minimizes the cross-entropy loss.

\begin{displaymath}
    L_{mlm} = \mathbb{E}_{(I,\hat{T})~D}H(y^{msk}, p^{msk}(I,\hat{T}))
\end{displaymath}

$\hat{T}$ is masked text, $P^{msk}(I,\hat{T})$ is the model's predicted probability for a masked token. 

Contrastive loss is used for alignment, which helps to ground the vision and text representations.

\subsection{RaSa: Relation and Sensitivity Aware}

Vision language models are very popular regions. Many studies have been published with the goal of binding the visual information with the text information. Although a variety of methods have been proposed, but learning a powerful multi-modal representation is still a challenging task. \cite{Bai2023RaSaRA} approaches to this task by two innovations: Relation-aware learning and Sensitivity-aware learning (\cite{Bai2023RaSaRA}).

\begin{figure}
    \includegraphics[width=\linewidth]{img/weak_positive_relation.png}
    \caption{Illustration of (a) two types of image-text pair in same class. The  }
\end{figure}

The authors remark that the text and image pair of the same ID have a strong positive pair and weak positive pair. Since the textual description is generated by a single image in the text-based person search dataset, the text will strongly correlate to the image, but won't always align well to other images of the same person. Previous methods did not take this intra-variation into account when training, and instead put equal weight for strong and weak positive pairs in learning representations. This led the model to overfitting due to the weak pairs.

To reduce the impact of noise interference from weak positive pairs, the authors introduced a Relation-Aware learning (RA) task. This task consists of a probabilistic Image-Text Matching (p-ITM) component and a Positive Relation Detection (PRD) component. The p-ITM is a variant of the commonly used ITM, designed to differentiate negative and positive pairs by probabilistically considering strong or weak positive inputs. Meanwhile, the PRD explicitly distinguishes between strong and weak positive pairs. In this framework, p-ITM focuses on the consistency between strong and weak positive pairs, whereas PRD emphasizes their differences, effectively acting as a regularization for p-ITM. By incorporating RA, the model can extract valuable information from weak positive pairs through p-ITM and reduce noise interference through PRD, ultimately achieving a balance.

Furthermore, improving the robustness of representations often involves learning invariant representations under a set of manually chosen transformations, referred to as insensitive transformations (\cite{caron2021unsupervisedlearningvisualfeatures}). While this approach is recognized, the authors went further by drawing inspiration from the recent success of equivariant contrastive learning (\cite{dangovski2022equivariantcontrastivelearning}). The authors explore sensitive transformations that would degrade performance if applied to learn transformation-invariant representations. Instead of maintaining invariance under insensitive transformations, they encourage the learned representations to be aware of sensitive transformations. 

To achieve this, the authors proposed a Sensitivity-Aware learning (SA) task. They use word replacement as the sensitive transformation and develop a Momentum-based Replaced Token Detection (m-RTD) pretext task. This task involves detecting whether a token originates from the original textual description or the replacement, as illustrated in Figure 1 (b). The closer the replaced word is to the original one (i.e., the more confusing the word), the more challenging the detection task becomes. Training the model to effectively solve this detection task is expected to enhance its ability to learn better representations.

The authors utilize Masked Language Modeling (MLM) to perform word replacement, leveraging the image and text contextual tokens to predict the masked tokens. Additionally, considering that a momentum model, which is a slow-moving average of the online model, can learn more stable representations than the current online model (\cite{grill2020bootstraplatentnewapproach}), the authors use MLM from the momentum model to generate more confusing words. 

Overall, MLM and m-RTD together form Sensitivity-Aware learning (SA), providing robust surrogate supervision for representation learning.

% In this section we will introduce the masking strategies for masked learning modeling. Existing pre-trained language models tries to have different masking strategies to enhance training efficiency and effectiveness. 
% Joshi, et al.  introduced SpanBERT which is designed to better represent and predict span of text.



\chapter{Proposed Methods}
% do i need this
% In this chapter, we introduce the proposed methods and experimental methodology. This research aims to address the question, "What would happen if we change the masking strategies?" While numerous strategies have been proposed to enhance the representation of text and images, there has been limited experimentation on altering the masking ratio in masked language modeling (MLM). This study primarily focuses on two main aspects:

% \begin{itemize}
%   \item The referenced model employs masked language modeling to train feature extraction, maintaining a constant masking ratio throughout the training process. This research proposes varying the masking ratio and introducing a time-variant component during training.
%   \item The referenced model includes a function known as momentum-based replace token detection, which also operates with a specific masking ratio. This study examines whether adjusting the masking ratio for this task, similar to the approach taken with MLM, improves prediction performance when both parameters are modified.
% \end{itemize}
%%%
% \color{WIP}
% The purpose of this experiment is to examine the hypothesis that time variant masking ratio affect the prediction on text based person retrieval.
% This is carried out by investigating the relation and sensitivity aware representation learning \cite{Bai2023RaSaRA} that investigated the better representation learning for image and text inputs by detecting the replaced tokens from the converted text and corresponding image inputs. The results showed significant improvements in prediction performance. To investigate further towards the masking ratio, \cite{wettig-etal-2023-mask} investigated that larger models should adopt a higher masking rate rather than masking 15\% of tokens conventionally. Another method from Dongjie Yang, et al, \cite{yang2023learningbettermaskingbetter} proposed time-variant masking ratio decay strategy and POS-tagging weighed masking. In results, the time variant masking decay method outperformed the time invariant masking ratio for F1 score on SQuAD performance during pre-training on BERT-large model. 

The purpose of this experiment is to examine the hypothesis that a time-variant masking ratio affects the prediction accuracy in text-based person retrieval tasks. We approach this by exploring the relationship between the sensitivity-aware representation learning, as investigated by Bai et al. (\cite{Bai2023RaSaRA}) in their work "RaSa," which focuses on improving representation learning for image and text inputs by detecting replaced tokens from the combined text and image inputs. Their results demonstrated significant improvements in prediction performance.

To delve deeper into the impact of masking ratios, (\cite{wettig-etal-2023-mask}) suggested that larger models benefit from adopting a higher masking rate rather than the  conventional 15\% of tokens. Additionally, (\cite{yang2023learningbettermaskingbetter}) proposed a time-variant masking ratio decay strategy along with POS-tagging weighted masking. Their findings indicated that the time-variant masking decay method significantly outperformed the time-invariant masking ratio in terms of F1 score on the SQuAD dataset during the pre-training of the BERT-large model.

Our experiment aims to build on these insights by evaluating the effects of different masking strategies on the performance of vision-language models, specifically in the context of text-based person retrieval. By systematically varying the masking ratio and employing time-variant masking techniques, we seek to determine the optimal approach for enhancing the alignment and interpretation of textual and visual data.

In this experiment, we investigate the effect of the performance towards the time variant masking ratio on replaced token detection task. To evaluate the performance, we will compute the feature similarity score for all image-text pairs. Then we take top-$k$ candidates and calculate their ITM score $s_{itm}$ for ranking. 

\section{Dataset}

\begin{figure}[htbp]
  \begin{center}
      \includegraphics[width=\linewidth]{img/cuhk_pedes.eps}
      \caption{Example sentence with corresponding image}
      \label{fig:cuhk_pedes}
  \end{center}
\end{figure}


Dataset we are using is CUHK-PEDES \cite{li2017personsearchnaturallanguage}. This is a large-scale dataset created to facilitate person search using natural language descriptions. It addresses the need for a dataset where persons are described in detail using natural language, enabling practical applications such as video surveillance. Key features of the dataset are shown as follows:
\begin{itemize}
  \item Large-scale: Dataset contains 40,206 images over 13,003 persons 
  \item Annotations: Each images is described with two sentences by independent annotators, in total of 80,412 sentence descriptions
  \item Source diversity: Images are sourced from a variety of existing person re-identification datasets, including CUHK03\cite{li2014deepreid}, Market-1501\cite{7410490}, SSM\cite{ssm}, VIPeR\cite{viper}, and CUHK01\cite{li2012human}.
\end{itemize}

The datasets contains image and corresponding natural language description as shown in Fig\ref{fig:cuhk_pedes}. Image source for CUHK-PEDES are from CUHK03, Market-1501, SSM, VIPER, and CUHK01. The annotations for each image are done by Amazon Mechanical Turk(AMT), which is a crowd workers that are employed to describe each image with two sentence. Each sentence include rich descriptions about person appearance, action poses, and interactions.


\section{Methods}
In this section, we introduce the experimental methods designed to evaluate the performance of vision-language models (VLMs). Recent state-of-the-art VLMs, such as the one described by Bai et al. (2023) in their work "RaSa," utilize an attention mechanism to extract and align features from both images and text using cross-modality encoders. Achieving optimal results necessitates not only robust feature extraction but also the development of sophisticated textual representations to enhance the interpretation of visual data. Therefore, our objective is to train the visual-language model (Bai et al., 2023) by employing diverse strategies for textual information representation. Specifically, we will investigate two distinct methods to determine their impact on model performance.


\section{Masking ratio for masked language modeling} 

\subsection{Introduction}
% explanation about what we do with masking ratio
In natural language pre-training, masked language modeling is utilized to understand the context of an input sentence. This same task is applied to recent vision-language models, where the masked token is predicted using both the image and the remaining text information. Although this method can yield good performance, the typical ratio for masking words in a sentence is fixed at 15\%. In our study, we aim to identify the optimal masking ratio by experimenting with time-invariant masking ratios ranging from 15\% to 40\%. Additionally, we will explore time-variant masking strategies based on previous findings to further enhance model performance.

\subsection{Procedure}

- モデルとパラメータの具体的な説明
- 始めにどの比率が効果的かの検証
- 次に時間変動での検証


To experiment on 

% explanation about what strategy we try to experiment
- what we are changing  explanation of parameter
- evaluation method for experiment
- the effect of the experiment 
- 

\section{masking ratio for replaced token detection}
RaSa utilizes this to create a task called sensitivity aware learning.
Sensitivity aware aims to make the model sensitive to specific transformations in the data, particularly textual changes. While many models aim to learn invariant representations that are robust to various data augmentations, RaSa takes it further by encouraging the model to detect and respond to sensitive transformations, such as word replacements in text. This is done using a Momentum-based Replaced Token Detection (M-RTD) task, where the model learns to detect whether a token in the text has been replaced. This sensitivity to changes enhances the robustness and discrimination power of the representations learned by the model.


Predicting more means the model learns from more training signals, so higher prediction rates boost the model performance. From another perspective, each prediction at each masked token leads to a loss gradient, which is averaged to optimize the weights of the model. Averaging across more predictions has a similar effect to increasing the batch size, which is proved to be beneficial for pre-training (Liu et al., 2019). 


corruption rate how much we mask the words from the text
prediction rate how much we predict the mask token 


what i did 
- change the mask 
- visualize the attention section

The structure of the algorithm is done as follows. 
- image encoder
  mlp 
- text encoder
  bert 

When we work on fine tuning, we change the masking ratio on the mlm process. 
the masking ratio will be changed as follows.
flat 
cosine curve 
random 

from these papers, bert scored higher accuracies when using 40\% on masking ratio. This is basically done in text information, 
so we would like to try it out in text based image retireval tasks. T

flat and cosine will work on the range of 40\% and 60\% 
the previous work has been done as 15\%, so we would compare the results with that for the discussion 

\section{Requirements specification}

- model
rasa

- dataset 
CUHK-dataset
ICFG-dataset

- epoch
30

- learning rate 
1e-5 to 1e-6

\section{Attention visualization}
To be able to see where the model is paying attention to, we would use the attention matrix in cross atttention module.



\section{Algorithm implementation}

for the 
% \chapter{Evaluation}

\section{Test suite}

\section{Experiments}

\section{Results}
% If I add all results to the Evaluation chapter, there's no need for a Results chapter.

% This will depend on how I structure the document once I start writing the experiments and results sections. For now, this stays empty.
\chapter{Discussion}

-- results line 
talk about my work
  check the numbers, use the visual grounding 
  - training loss Graph
  the results shows the loss prd, masking ratio, and loss mlm.
  loss mlm correlates to the masking ratio, but prd decrease the same way for all 
  - evaluation 
  the r1~r10 does not change at all 

-- discussion line
discuss about the prediction 
- my work didn't change at all 
- for mlm, this method is major task for NLP and VLM models
- previous method did an analysis in pre-training steps, which made significant difference because it's the part where the model actually learns how to achieve information from the modality
- in fine-tuning, they're trained to generate the answers which corresponds to specific tasks
- it's just fitting the model to specific method which does not effect that much to the actual model knowledge, that's why the fine-tuning part is more important to have better training task that really fits to the task you want the model to solve <- this requires good reference to explain. use the rasa results and the results i have
  also use the visual grounding as well 



\chapter{Conclusions}
In this study, we explored the impact of various masking strategies on the performance and attention mechanisms of \acrfull{vlm} in the context of text-based person retrieval tasks using the CUHK-PEDES dataset. Our experiments revealed that while different masking ratios can influence specific performance metrics, such as \acrfull{map}, the overall effect on the model's accuracy and robustness remains limited.

The analysis of attention patterns showed that at lower masking ratios, the model tends to focus on finer details within the image, whereas higher masking ratios lead to broader attention across larger image regions. However, a key observation was that the attention to image regions indicated by corresponding words in the text remains stable across different masking strategies. This consistency underscores the model's ability to maintain robust and meaningful connections between textual descriptions and visual features, even when varying the level of masked data during training.

These findings contribute to a deeper understanding of how \acrshort{vlm} process and align multi-modal data, highlighting the importance of attention mechanisms in maintaining consistency and robustness in image-text retrieval tasks. Future work could extend this research by exploring more sophisticated masking techniques or applying these insights to other datasets and domains. Additionally, improving model interpretability and further enhancing the robustness of \acrshort{vlm} could lead to more reliable and effective applications in real-world scenarios.



\appendix
% \chapter{Prisma Automator}\label{chap:Prisma Automator}

The ``Prisma Automator'' program was developed to automate the initial steps outlined in the PRISMA2020 Statement\cite{prismastatement}. This process, typically done manually, involves formulating search strings, retrieving document metadata, and filtering results — tasks that become increasingly repetitive with more keyword combinations. The program aims to simplify user interaction by handling these steps, requiring only input of desired keywords and subsequent monitoring of the resulting document pool.

Comprising two classes, ``Splitter'' and ``Collector'', Prisma Automator facilitates the generation of search strings (splits) and interacts with the Scopus API to retrieve, clean, and save results locally. Both classes offer streamlined functionality through the ``split()'' method in Splitter and the ``run()'' method in Collector, but users have the flexibility to employ other methods or customize functionality as needed. 

Prisma Automator is an open-source project available at \url{https://github.com/Fabulani/prisma-automator}.


\chapter{BERT}

BERT (Bidirectional Encoder Representations from Transformers) is a transformer-based model designed for natural language processing (NLP) tasks. Developed by researchers at Google and introduced in the paper "BERT: Pre-training of Deep Bidirectional Transformers for Language Understanding" in 2018, BERT has significantly advanced the state of the art in NLP by providing a powerful pre-trained language representation model. Here's a detailed explanation of BERT:

 Key Concepts of BERT

1. **Bidirectionality**:
   - Traditional language models, like the original GPT, are unidirectional, meaning they predict the next word in a sequence based only on the words that come before it. BERT, however, is bidirectional. It considers both the left and right context when predicting a word, which allows it to better understand the context and semantics of the text.

2. **Transformers**:
   - BERT is built upon the transformer architecture, specifically utilizing the encoder part of the transformer. The transformer’s self-attention mechanism allows BERT to focus on different parts of the input text to understand the relationships between words and phrases better.

 Pre-training and Fine-tuning

BERT involves two main steps: pre-training and fine-tuning.

1. **Pre-training**:
   - **Masked Language Modeling (MLM)**: During pre-training, some percentage of the input tokens are randomly masked, and the model is trained to predict the original tokens based on the context provided by the unmasked tokens. This helps BERT learn bidirectional representations of the text.
   - **Next Sentence Prediction (NSP)**: This task involves predicting whether a given sentence B is the actual next sentence that follows a given sentence A in the corpus. This helps BERT understand the relationships between sentences.

2. **Fine-tuning**:
   - After pre-training, BERT can be fine-tuned on a specific NLP task such as question answering, sentiment analysis, or named entity recognition. During fine-tuning, the pre-trained BERT model is further trained on a labeled dataset for the specific task, adjusting its weights to optimize performance on that task.

 BERT Variants

Several variants of BERT have been developed to cater to different needs:

1. **BERT Base**: Consists of 12 layers (transformer blocks), 768 hidden units, and 12 attention heads.
2. **BERT Large**: Consists of 24 layers, 1024 hidden units, and 16 attention heads.
3. **DistilBERT**: A smaller, faster, and lighter version of BERT, retaining 97% of BERT's language understanding while being 60% faster.
4. **RoBERTa**: Robustly optimized BERT approach, which improves on BERT by training with more data and longer sequences.
5. **ALBERT**: A lite version of BERT, which reduces the model size while maintaining performance by sharing parameters across layers.

 Applications of BERT

BERT has been applied to a wide range of NLP tasks, including:

1. **Question Answering**: BERT can be fine-tuned on datasets like SQuAD to understand and answer questions based on given texts.
2. **Sentiment Analysis**: Determining the sentiment expressed in a piece of text.
3. **Named Entity Recognition (NER)**: Identifying and classifying named entities (e.g., people, organizations, locations) within a text.
4. **Text Classification**: Classifying texts into different categories.
5. **Language Translation**: Assisting in translating text from one language to another.
6. **Text Summarization**: Generating concise summaries of longer texts.

 Impact of BERT

BERT has had a transformative impact on NLP by providing a robust, pre-trained model that can be fine-tuned for a variety of tasks with relatively little labeled data. This has significantly lowered the barrier to achieving state-of-the-art performance on many NLP benchmarks and has spurred further research and development in transformer-based models.

In summary, BERT's introduction of bidirectional context representation, coupled with its robust pre-training on large corpora and the flexibility of fine-tuning for specific tasks, has made it a cornerstone in modern NLP, influencing a wide array of applications and subsequent model developments.


% THIS IS AN EXAMPLE OF USING CITATIONS:
% Graph generators are important \citep{metzler18random}.
% \citet{kalofolias18from} discuss sets of redescriptions.


\printnoidxglossaries

%% Next comes the references
\printbibliography[heading=bibintoc]

\backmatter % Do not remove!
%% Possible appendices come here
\end{document}
\endinput
%%
%% End of file `minimal_imlex.en.tex'.