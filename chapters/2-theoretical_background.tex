\chapter{Theoretical Background}
This section will describe the theoretical background for the techniques used for text encoder, image encoder, and cross attention methods. 
\section{Transformer}
A transformer is a type of deep learning model architecture that has revolutionized the field of natural language processing (NLP) and has been extended to various other domains such as computer vision and speech recognition. The transformer model was introduced in the paper "Attention is All You Need" by Vaswani et al. in 2017. Here's an overview of its key components and how it works:

 Key Components of a Transformer

1. **Self-Attention Mechanism**:
   - **Self-attention** (or scaled dot-product attention) allows the model to weigh the importance of different words in a sentence relative to each other. It enables the model to focus on relevant parts of the input sequence when generating an output.
   - For each word, self-attention calculates a weighted sum of all the words in the sentence, where the weights are determined by the relevance of the words to each other.

2. **Multi-Head Attention**:
   - Multi-head attention allows the model to jointly attend to information from different representation subspaces at different positions. Instead of performing a single attention function, the transformer splits the input into multiple heads and performs attention in parallel, then concatenates the results.

3. **Positional Encoding**:
   - Since transformers do not have a built-in notion of the order of elements in a sequence, positional encoding is added to the input embeddings to provide information about the position of each word in the sequence.

4. **Feed-Forward Neural Network**:
   - Each position in the sequence is processed independently through a feed-forward neural network. This is applied identically to each position, providing additional processing power to the model.

5. **Layer Normalization and Residual Connections**:
   - Layer normalization helps in stabilizing and accelerating the training process.
   - Residual connections (or skip connections) allow gradients to flow more easily through the network, which helps in training deep networks.

 Architecture Overview

A transformer model is composed of an encoder and a decoder, both of which are stacks of identical layers:

- **Encoder**: Each layer in the encoder consists of two sub-layers: a multi-head self-attention mechanism and a position-wise fully connected feed-forward network. The input to the encoder is a sequence of word embeddings, along with positional encodings.
  
- **Decoder**: Each layer in the decoder has three sub-layers: a multi-head self-attention mechanism, a multi-head attention mechanism over the encoder's output, and a position-wise fully connected feed-forward network. The decoder generates the output sequence by attending to both the previously generated tokens and the encoder's output.

 Advantages of Transformers

1. **Parallelization**:
   - Unlike recurrent neural networks (RNNs), transformers do not rely on sequential processing, allowing for much more parallelization during training, which significantly speeds up the process.

2. **Handling Long Dependencies**:
   - The self-attention mechanism allows transformers to handle long-range dependencies more effectively than RNNs and LSTMs, which can struggle with distant dependencies due to vanishing gradients.

3. **Scalability**:
   - Transformers have shown to scale well with the availability of large datasets and computational resources, leading to significant improvements in performance as the model size increases.

 Applications

- **Natural Language Processing**: Tasks such as translation, text summarization, and sentiment analysis.
- **Computer Vision**: Vision Transformers (ViTs) have been used for image classification and object detection.
- **Speech Recognition**: Transformers have been applied to model speech data and transcribe audio recordings.

 Popular Transformer Models

- **BERT (Bidirectional Encoder Representations from Transformers)**: Pre-trained on large text corpora to generate contextual embeddings for NLP tasks.
- **GPT (Generative Pre-trained Transformer)**: A series of models designed for generating coherent and contextually relevant text.
- **T5 (Text-To-Text Transfer Transformer)**: Converts all NLP tasks into a text-to-text format, providing a unified framework for diverse tasks.

In summary, the transformer model has become the foundation for many state-of-the-art approaches in NLP and beyond due to its ability to handle complex dependencies and its efficiency in training and inference.

% \section{ALBEF}
% The paper "Align before Fuse: Vision and Language Representation Learning with Momentum Distillation" by Junnan Li et al. from Salesforce Research presents a new framework for vision-and-language pre-training (VLP) called ALBEF (Align BEfore Fuse). Here is an overview of the main concepts and innovations introduced in the paper:

%  Key Concepts and Innovations:

% 1. **Problem with Existing Methods**:
%    - Most existing VLP methods use a multimodal encoder to jointly model visual tokens (image features) and word tokens (text features).
%    - These methods face challenges because the visual and word tokens are unaligned, making it difficult for the encoder to learn interactions between them.
%    - Additionally, these methods often rely on expensive object detectors and high-resolution images, which can be computationally intensive and require extensive annotations.

% 2. **ALBEF Framework**:
%    - **Alignment before Fusion**: ALBEF first aligns image and text representations before fusing them through cross-modal attention. This alignment is achieved using a contrastive loss, which helps to ground the vision and language representations.
%    - **Detector-Free Encoding**: ALBEF uses a detector-free image encoder and a text encoder to independently process images and text before combining them with a multimodal encoder.
%    - **Momentum Distillation (MoD)**: To handle noisy web data, ALBEF introduces momentum distillation, a self-training method that uses pseudo-targets generated by a momentum model (a slowly evolving teacher model) to guide the learning process.

% 3. **Pre-Training Objectives**:
%    - **Image-Text Contrastive (ITC) Learning**: This loss aligns the representations from the unimodal encoders (image and text encoders) before they are fused.
%    - **Masked Language Modeling (MLM)**: Similar to BERT, this objective involves predicting masked words in the text using both image and text context.
%    - **Image-Text Matching (ITM)**: This objective determines whether a given image and text pair is a match, incorporating contrastive hard negative mining to improve the learning process.

% 4. **Theoretical Analysis**:
%    - The paper provides a theoretical perspective, showing that ALBEF maximizes mutual information between different views of an image-text pair. Different training tasks (ITC, MLM) are interpreted as generating different views, enhancing the robustness and generalization of the learned representations.

% 5. **Performance**:
%    - ALBEF achieves state-of-the-art performance on various downstream vision-language tasks, including image-text retrieval, visual question answering (VQA), and visual reasoning (NLVR2).
%    - It outperforms existing methods that use much larger pre-training datasets and offers faster inference speeds.

% 6. **Implementation Details**:
%    - ALBEF employs a 12-layer visual transformer (ViT-B/16) as the image encoder and a 6-layer transformer for both the text and multimodal encoders.
%    - The pre-training dataset consists of large-scale image-text pairs, and the model is fine-tuned on specific downstream tasks.

%  Summary
% ALBEF introduces a novel approach to vision-and-language representation learning by aligning image and text representations before fusing them and leveraging momentum distillation to improve learning from noisy data. This framework not only addresses the limitations of previous methods but also achieves superior performance on multiple vision-language tasks.


\section{GradCAM}
Ramprasaath R. Selvaraju et al. presents a method for providing visual explanations for the decisions made by Convolutional Neural Networks (CNNs). This method, called Gradient-weighted Class Activation Mapping (Grad-CAM), generates a coarse localization map that highlights important regions in an image, which are crucial for predicting a specific concept.

Utilizes gradients flowing into the final convolutional layer to produce a localization map.
Applicable to various CNN architectures, including those with fully-connected layers (like VGG), models for structured outputs (like image captioning), and multi-modal inputs (like visual question answering).
Combining with Fine-Grained Visualizations:

Grad-CAM is combined with pixel-space gradient visualizations to create Guided Grad-CAM, which offers high-resolution, class-discriminative visualizations.
Applications and Insights:

Applied to image classification, image captioning, and visual question answering models.
Helps identify failure modes and dataset biases, providing insights into model predictions.
Demonstrates that even non-attention-based models can localize relevant input regions.
Human Studies:

Shows that Grad-CAM explanations help users build appropriate trust in model predictions.
Helps untrained users discern between stronger and weaker networks, even when their predictions are identical.
Benefits of Grad-CAM:

Does not require architectural changes or re-training of the network.
Outperforms previous methods in terms of weakly-supervised localization tasks and model faithfulness.
The paper emphasizes the importance of interpretability in AI systems, aiming to bridge the gap between model accuracy and transparency. By providing clear visual explanations, Grad-CAM enhances the trustworthiness and usability of deep learning models in practical applications.

\section{GradCAM++}
 Detailed Explanation of Grad-CAM++
Grad-CAM++ is an extension of the Grad-CAM method that enhances the localization of objects in images and provides more precise visual explanations. Here are the key components and improvements introduced by Grad-CAM++:

 1. **Background and Motivation**
- **Grad-CAM Limitations**: While Grad-CAM is effective at highlighting important regions in an image, it has some limitations, such as difficulty in handling multiple instances of the same class within an image and partial localization of objects.
- **Need for Improvement**: To address these issues, Grad-CAM++ introduces a more refined approach that can better capture the spatial importance of different regions in an image.

2. **Methodology**
- **Weighted Combination of Gradients**: Grad-CAM++ uses a weighted combination of positive partial derivatives of the target class score with respect to the feature maps of the last convolutional layer. This allows for more accurate and detailed localization of relevant regions.
  - **Mathematical Derivation**: The method involves calculating the importance of each pixel in the feature maps by considering the gradients and their spatial distribution.
  - **Positive Partial Derivatives**: By focusing on positive gradients, Grad-CAM++ ensures that only the regions positively contributing to the class prediction are highlighted.

- **Pixel-Wise Weighting**: Unlike Grad-CAM, which uses global average pooling of gradients, Grad-CAM++ applies pixel-wise weighting, giving more precise importance scores to different parts of the feature maps.
  - **Equation for Grad-CAM++**: 
    \[
    L_{\text{Grad-CAM++}}^c = \text{ReLU} \left( \sum_k \alpha_k^c A^k \right)
    \]
    where \(\alpha_k^c\) is the weight for the \(k\)-th feature map \(A^k\), and it is derived from the positive partial derivatives.

 3. **Evaluation Metrics**
- **Faithfulness to Model**: Grad-CAM++ proposes new metrics to evaluate how faithfully the visual explanations represent the underlying model's decision-making process.
  - **Localization Metrics**: These metrics assess the accuracy of object localization, helping to objectively compare the performance of Grad-CAM++ against other methods.

 4. **Experimental Results**
- **Object Localization**: Grad-CAM++ consistently outperforms Grad-CAM and other state-of-the-art methods in tasks requiring object localization.
  - **Weakly Supervised Localization**: The method is particularly effective in scenarios where only image-level labels are available, demonstrating superior localization capabilities.

- **Diverse Applications**: Grad-CAM++ is tested on a variety of tasks beyond traditional image classification, including:
  - **Image Captioning**: Where the method helps identify regions relevant to generating specific parts of a caption.
  - **3D Action Recognition**: Where it aids in understanding which frames or regions are crucial for recognizing actions in videos.

 5. **Human Studies**
- **User Trust and Interpretability**: Human studies show that users find explanations generated by Grad-CAM++ more trustworthy and easier to understand compared to those generated by Grad-CAM.
  - **Qualitative Assessments**: Participants in the studies preferred the detailed and accurate visual explanations provided by Grad-CAM++.

Contributions and Impact
- **Improved Explanations**: Grad-CAM++ provides more detailed and accurate visual explanations, enhancing the interpretability of CNNs.
- **Broader Applicability**: The method is versatile and applicable to a wide range of tasks, demonstrating robustness across different domains.
- **Trust and Transparency**: By offering clearer and more precise visual explanations, Grad-CAM++ helps build greater trust in AI systems and aids in debugging and understanding model behavior.

In summary, Grad-CAM++ represents a significant step forward in the field of explainable AI, particularly in providing high-quality, interpretable visual explanations that are essential for both understanding and trusting deep learning models.

\section{ScoreCAM}
Background
Understanding how convolutional neural networks (CNNs) make decisions is crucial for increasing their transparency and trustworthiness. Several methods have been developed for this purpose, including gradient-based methods, perturbation-based methods, and class activation mapping (CAM)-based methods.

Existing Methods
Gradient-Based Methods:

These methods highlight image regions that significantly influence the prediction by backpropagation the gradient of the target class to the input layer.
Examples include Saliency Maps and Grad-CAM. However, these methods often produce noisy and low-quality visual explanations.
Perturbation-Based Methods:

These involve perturbing the input image and observing changes in the model's predictions.
While these methods can be effective, they often require additional regularization and can be time-consuming.
CAM-Based Methods:

These create visual explanations by linearly combining activation maps from convolutional layers.
CAM is architecture-sensitive and requires a global pooling layer, which is not always present in all models.
Grad-CAM and Grad-CAM++ generalize CAM to work with models without a global pooling layer by using gradient information to weight the activation maps.
Score-CAM
The authors propose Score-CAM as a new method to address the limitations of existing gradient-based and CAM-based methods:

Gradient-Free: Unlike Grad-CAM, Score-CAM does not rely on gradient information, which can be noisy and prone to saturation issues.
Forward Passing Score: Score-CAM determines the weight of each activation map by its contribution to the target class score, assessed through a forward pass of the input.
Linear Combination: The final visual explanation is obtained by a linear combination of the weighted activation maps, which highlights the regions of the input image most relevant to the target class.
Methodology
Increase of Confidence: This concept quantifies the importance of each activation map by measuring the increase in the target class score when the corresponding activation map is used as input.
Implementation:
The method involves masking the input image with upsampled activation maps and measuring the change in the target score.
This change represents the importance of the activation map, which is then used as its weight.
Contributions and Evaluation
Fairness and Performance:

Score-CAM is evaluated on recognition tasks using metrics like Average Drop, Average Increase, Deletion curve, and Insertion curve, showing better evidence of the target class compared to other methods.
Qualitative evaluations also show better visualization and localization performance.
Debugging Tool:

The method is effective for analyzing and debugging model misbehaviors, providing insights into why a model made a specific prediction.



Visual Grounding aims to localize the region in an image that corresponds to a specific textual description. We study the weakly-supervised setting, where no bounding box annotations are available. We perform experiments on the RefCOCO+ [56] dataset, and fine-tune the model using only imagetext supervision following the same strategy as image-text retrieval. During inference, we extend Grad-CAM [9] to acquire heatmaps, and use them to rank the detected proposals provided by [53].

visualizations on the self-attention maps in the last layer of the image encoder, where the gradients are acquired by maximizing the image-text similarity sitc. The ALBEFitm variant computes Grad-CAM on the cross-attention maps in the 3rd layer of the multimodal encoder (which is a layer specialized in grounding), where the gradients are acquired by maximizing the image-text matching score sitm. Figure 4 provides a few visualizations.


Visual Grounding. We conduct experiments on the RefCOCO+ dataset [56], which is collected using a two-player ReferitGame [62]. It contains 141,564 expressions for 19,992 images from COCO training set. Strictly speaking, our model is not allowed to see the val/test images of RefCOCO+, but it has been exposed to those images during pre-training. We hypothesize that this has little effect because these images only occupy a very small portion of the entire 14M pre-training images, and leave it as future work to decontaminate the data. During weakly-supervised fine-tuning, we follow the same strategy as image-text retrieval except that we do not perform random cropping, and train the model for 5 epochs. During inference, we use either sitc or sitm to compute the importance score for each 16 × 16 image patch. For ITC, we compute Grad-CAM visualizations on the self-attention maps w.r.t the [CLS] token in the last layer of the visual encoder, and average the heatmaps across all attention heads. For ITM, we compute Grad-CAM on the cross-attention maps in the 3rd layer of the multimodal encoder, and average them scores across all attention heads and all input text tokens. Quantitative comparison between ITC and ITM is shown in Table 5. Figure 7 shows the qualitative comparison. Since the multimodal encoder can better model image-text interactions, it produces better heatmaps that capture finer-grained details.


\section{Visual Grounding}
% ai
The process of connecting certain textual passages to certain areas of an image is known as "visual grounding." This is necessary for many applications, such as picture-text retrieval, image captioning, and visual question answering. Visual grounding in the context of Visual Language Models guarantees that the model can comprehend and produce language in respect to visual content appropriately. Effective visual grounding is achieved by utilizing a number of methods and approaches. Using attention mechanisms, which let the model concentrate on pertinent areas of the image while processing the text, is one popular method. These methods produce attention maps, which can be used to visualize how the model links certain areas of the image to particular words or phrases in the text.

Grad-CAM generates a coarse localization map highlighting important regions in an image for predicting a particular class. This is done by computing the gradient of the class score with respect to the feature maps of a convolutional layer. The resulting gradient information is then used to weight the feature maps, producing a heatmap that shows the areas of the image most relevant to the prediction.

Grad-CAM++:
Grad-CAM++ is an extension of Grad-CAM that provides better localization and more accurate heatmaps. It addresses some of the limitations of Grad-CAM, particularly in cases where multiple instances of the same class are present in the image. Grad-CAM++ achieves this by using higher-order derivatives and a more sophisticated weighting scheme for the feature maps, leading to finer and more detailed heatmaps.
% ai

To understand how each text, such as word or phrase, are linked to the image, visual grounding are utilized. Visual grounding utilize text and image information to specific region in an image. This method guarantees that the model can comprehend and produce language in respect to visual content appropriately. Effective visual grounding have been achieved by vairous methods. For VLM especially, the model applies attention mechanisms which can be used to visualize how the model links certain areas of the image to particular words or phrases in the text.
To achieve visual grounding with attention, commonly GradCAM(\cite{gradcam}) is used. This method is originally proposed for visual explanation in convolutional neural network (CNN). GradCAM utilizes gradients flowing into the convolutional layer to produce a localization map. The resulting gradient information is then used to weight thte feature maps, which produces a heatmap to show the areas of the image that is the most relevent to the prediction. The benefits of Grad-CAM is that the visualization is done by the gradient information, which does not require architecutral changes or re-training of the network.

Grad-CAM outperformed the previous method in terms of wealkly-supervised localization tasks and model faithfulness, but relying on 

\subsection{Grad-CAM}
Ramprasaath R. Selvaraju et al. present a method for providing visual explanations for the decisions made by Convolutional Neural Networks (CNNs). This method, called Gradient-weighted Class Activation Mapping (Grad-CAM), generates a coarse localization map that highlights important regions in an image crucial for predicting a specific concept.

Grad-CAM utilizes gradients flowing into the final convolutional layer to produce a localization map. It is applicable to various CNN architectures, including those with fully-connected layers (like VGG), and models for structured outputs. When combined with pixel-space gradient visualizations, Grad-CAM creates Guided Grad-CAM, which offers high-resolution, class-discriminative visualizations.

Following steps enables to obtain the class-discriminative localization map $L^c_{Grad-CAM} = \in\Re^{u \times v}$ of class $c$ from the image width $u$ and height $v$. 

First, we would define the neuron importance weights $\alpha^c_k$. calculate the gradient of the score for class c, $y_c$(before softmax), with respect to the feature map activations $A_k$ of a convolutional layer, i.e., $\frac{\delta y^c}{\delta A^k}$. 
 
\begin{displaymath}
    L^c_{Grad-CAM} = ReLU(\sum_k\alpha^c_kA^k)
\end{displaymath}

One significant advantage of this method is its class-discriminative nature. Grad-CAM can highlight specific regions relevant to a particular class, making it useful for understanding model predictions. Another advantage is the high resolution of the visualizations. By combining Grad-CAM with existing fine-grained visualization techniques like Guided Backpropagation, the authors create Guided Grad-CAM, which provides detailed and class-discriminative visual explanations.

\subsection{Grad-CAM++}
Grad-CAM however, it has some limitations, such as difficulty in handling multiple instances of the same class within an image and partial localization of objects. To address these issues, Grad-CAM++ introduces a more refined approach that can better capture the spatial importance of different regions in an image. 