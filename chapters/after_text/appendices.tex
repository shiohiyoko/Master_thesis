\chapter{Prisma Automator}\label{chap:Prisma Automator}

The ``Prisma Automator'' program was developed to automate the initial steps outlined in the PRISMA2020 Statement\cite{prismastatement}. This process, typically done manually, involves formulating search strings, retrieving document metadata, and filtering results — tasks that become increasingly repetitive with more keyword combinations. The program aims to simplify user interaction by handling these steps, requiring only input of desired keywords and subsequent monitoring of the resulting document pool.

Comprising two classes, ``Splitter'' and ``Collector'', Prisma Automator facilitates the generation of search strings (splits) and interacts with the Scopus API to retrieve, clean, and save results locally. Both classes offer streamlined functionality through the ``split()'' method in Splitter and the ``run()'' method in Collector, but users have the flexibility to employ other methods or customize functionality as needed. 

Prisma Automator is an open-source project available at \url{https://github.com/Fabulani/prisma-automator}.

