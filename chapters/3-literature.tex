\chapter{Related Studies }
In this chapter we will briefly introduce the basics of person retrieval methods. To match two different modality data, recent studies uses attention based deep learning model. This way, the model could catch the feature from both modality and combine the same meanings. So we will mainly describe about the attention methods.


% ------------------------------------------------------------ %
\section{Research Methodology}
In the methodology section, the we first delves into the existing literature, drawing from a paper accessible through the platform "Paper with Code." This platform typically provides research papers along with their associated code implementations. The chosen paper appears to be selected based on its prominence, likely measured by its reported accuracy or success in the field.
Following the identification of the primary paper, the researcher conducts a thorough review of its content, focusing particularly on aspects related to methodology. This involves understanding the proposed techniques, algorithms, and approaches presented in the paper to achieve high accuracy in the context of text-based person searches. The aim is to comprehend the nuances of the existing methodology and identify the key factors contributing to its success.
In addition to the primary paper, the researcher examines two other papers that exhibit a significant difference in accuracy. This comparative analysis is valuable for gaining insights into different approaches within the field. The choice of these additional papers may be strategic, aiming to capture diverse perspectives or methodologies, especially if there is a notable contrast in their reported accuracy metrics.
The researcher likely scrutinizes the methodologies of these selected papers, comparing and contrasting them with the primary paper. This comparative analysis helps identify the strengths and weaknesses of different approaches, shedding light on potential areas of improvement or innovation for the current research.
Overall, the methodology involves a comprehensive exploration of relevant literature, with a focus on the primary paper selected from "Paper with Code." The intent is to understand the methodologies employed in achieving high accuracies and to leverage insights from other papers with varying performance metrics. 

However, if only paperwithcode is used, the information obtained is limited and biased. To eliminate this bias, we decided to use scopus to search a wider range of papers by keyword search.

\subsection*{Identification}

The following research question was defined:

\bigskip
\textit{``How does the model compares with the text information with the image information''}
\bigskip



From this research question, four main keywords that sufficiently explain the topic were used: person retrieval and vision language pre-training.
Furthermore, synonyms and related terms were associated to these keywords to form keyword groups as follows:

\begin{itemize}
    \item person retrieval:
    \begin{itemize}
        \item person;
        \item person detection;
        \item person search.
    \end{itemize}
    \item vision language pre-training:
    \begin{itemize}
        \item VLP;
        \item text based;
        \item text.
    \end{itemize}
\end{itemize}


From the keywords, we had a keyword search on scopus from the search strings as follows:

\begin{itemize}
    \item ( "person retrieval" OR "person" OR "person detection" OR "person search" ) AND ( "vision language pre-training" OR "VLP" OR "text based" OR "text" ).
\end{itemize}

The Scopus search yielded a total of $20170$ documents. Within this result, we set the subject area to Computer Science, document type to article and conference paper, language to only english, and set the open access to all open access. With this filters, $862$ articles were found. 

\subsection{Screening}

Various factors were taken into account for the exclusion of documents:
\begin{enumerate}
    \item problem and goal were too different (e.g., building new hardware, analysis of leaf reflectance);
    \item not sufficiently related to this work (e.g., focused on hyperspectral );
    \item duplicates that were not automatically detected and excluded.
\end{enumerate}

% ------------------------------------------------------------ %

\section{Vision-language models}
before transformer
transformer came, 
with the transformer, we can have the model with better precision
the sota model today we have rasa 
this requires model to be sensitive to the text 

albef -> rasa

Vision-Language models are a model that combines both the vision and language modalities and enables to process both information. 
Take, for example, the task of zero-shot image classification. We’ll pass an image and a few prompts like so to obtain the most probable prompt for the input image.
To predict the probable prompt, the model needs to understand both input image and the text prompts. To understand those modalities, the model will have separate or fused encoders for both vision and language.


\subsection{Learning Strategies}
Contrastive learning aims to map input images and texts to the same feature space such that the distance between the embeddings of image-text pairs is minimized if they match or maximized if they don’t. This method is a commonly used pre-training objectives for vision models and proven to be a highly effective for vision-language models as well. \cite{radford2021learning} uses this learning strategy with a cosine distance betweent the text and image embeddings. For pre-training methods requires large datasets to train, so most of the times, they use image and corresponding caption from the internet to train. This way, the model can train with large data, but in the other hand the image and caption sometimes does not correlate. To deal with this problem, ALIGN\cite{jia2021scaling} and DeCLIP\cite{li2022supervision} designed their own distance metrics.


\subsection{BERT}


\subsection{ALBEF}

\subsection{RaSa}

Vision language models are very popular regions. Many studies have been published to be able to bind the visual information with the text information. 

key concept and innovation
relation aware
既存のイメージテキストマッチングではペアの正負のみ判別している。


この作者は同じIDの画像テキストペア1と画像テキストペア2では強い相関のある特徴と弱い相関のある特徴が存在する。筆者らはこの問題に対して相関の強いペアをより強く学習し、弱いペアを小さい値pwだけ学習させることにした。これにより強い相関のあるペア同士をより学習するとともに弱いペアに生じるノイズを抑止することが可能となる。これにより弱いペアのノイズから過学習を起こさないようにし、学習することが可能である。

sensitivity aware

次に著者らは文章中の表現に敏感になるように課題を足しました。

この能力を達成させるために筆者らはテキストからランダムに単語を選択し、親しい単語に変換し、モデルに度の単語を変換させたのかを判断させる課題を提案しました。
このシステムでは筆者らはテキストの表現を理解するためにmasked language modelタスクを利用し、より親しい


\section{Masking Strategies}
Methods for masking have many different varieties, so from those methods, I pick to change the masking ratio.