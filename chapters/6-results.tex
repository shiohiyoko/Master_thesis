% If I add all results to the Evaluation chapter, there's no need for a Results chapter.

% This will depend on how I structure the document once I start writing the experiments and results sections. For now, this stays empty.
\chapter{Discussion}

-- results line 
talk about my work
  check the numbers, use the visual grounding 
  - training loss Graph
  the results shows the loss prd, masking ratio, and loss mlm.
  loss mlm correlates to the masking ratio, but prd decrease the same way for all 
  - evaluation 
  the r1~r10 does not change at all 

-- discussion line
discuss about the prediction 
- my work didn't change at all 
- for mlm, this method is major task for NLP and VLM models
- previous method did an analysis in pre-training steps, which made significant difference because it's the part where the model actually learns how to achieve information from the modality
- in fine-tuning, they're trained to generate the answers which corresponds to specific tasks
- it's just fitting the model to specific method which does not effect that much to the actual model knowledge, that's why the fine-tuning part is more important to have better training task that really fits to the task you want the model to solve <- this requires good reference to explain. use the rasa results and the results i have
  also use the visual grounding as well 


