\chapter{Introduction}
Person re-identification is a crucial topic with applications in service robotics and security. The goal is to extract features from an input source and compare them against a large dataset to identify the corresponding individual. Traditional methods predominantly rely on image-based inputs, but these methods face significant challenges due to intra-class variations caused by different viewing angles, environmental changes, and color variations.

To address these issues, with advances in learned models of natural language processing, vision language models are used. There are few methods to train the vision-language models. 
Contrastive training is a common strategy that utilizes pairs of positive and negative examples. In this method, the VLM is trained to produce similar representations for positive pairs and different representations for negative pairs. Another strategy is masking, where the VLM learns to reconstruct missing patches from an unmasked text caption. Similarly, masking words in a caption allows the VLM to reconstruct those words using an unmasked image.

While many approaches use intermediate representations or partial reconstructions, generative VLMs are designed to generate entire images or lengthy captions. Due to their complexity, these models are often the most expensive to train. VLMs with pretrained backbones frequently employ open-source LLMs like Llama to establish a mapping between a pre-trained image encoder and the LLM. It's important to note that these paradigms are not mutually exclusive; many approaches combine contrastive, masking, and generative criteria.

Especially for person retrieval tasks, VLM are trained with combination of contrastive learning and masking. It utilized BERT as the text encoder and Vision Transformer as the image encoder. Typically, the text encoder employs a fixed masking ratio for the masked language modeling (MLM) tasks. This thesis investigates the impact of varying the masking ratio during the training of the visual language model, aiming to understand how these changes affect the model's performance.

\section{Motivation}
The increasing potential for robots to collaborate with humans in various work environments necessitates the development of intuitive communication methods. The most natural way for humans to interact is through text or spoken information. By enabling robots to understand and respond to these forms of communication, we can significantly enhance human-robot collaboration.

Such advancements could address critical issues like manpower shortages by allowing robots to seamlessly integrate into human workforces. To achieve this level of interaction, it is essential to effectively connect and interpret information from both text and images. This requires developing a cross-modal model that can map and recognize features across these different types of data. By focusing on creating such a model, we aim to bridge the gap between text and image information, facilitating more intuitive and efficient human-robot collaboration.

\section{Objectives}
- intra difference
- camera distortion
- text information is not enough
- attention differences

we will tackle through the attention differences between text and image information
previous methods worked on getting better multi-modal representation 
many approaches have been produced, but when we look into mlm models, they do not change the masking ratio. there are studies that had a question about this, but it did not go beyond to person retrieval models. we would like to look into this and see if the accuracy will change.

\section{Document structure}

what i am doing 
trying to improve the current person retrieval methods

sota methods rely on attention mechanism to match image and text 
important role in i2t is to get the text information and image attention correctly 
sota model uses sensitivity aware well 
sensitivity aware requires to find the changed text from the input text 
to have the model extract the information very well, we need to find the best masking ratio to improve the model
sensitivity aware uses 
